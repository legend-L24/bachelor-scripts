% Copyright (c) 2014,2016,2018 Casper Ti. Vector
% Public domain.

\chapter{结论和展望}

本人提出了一种具有通用性的晶体学碰撞检测算法框架,按照此框架让 sweep and prune%
(SAP)算法能以 $O(n\log n)$ 的时间复杂度检测晶胞中的成键关系,并提出了一种
在小晶胞中也适用的利用等效点系对称性极大降低碰撞检测复杂度的算法,最后基于以上
算法提出了一种针对晶胞中原子重叠状况的评估函数。利用以上结果,我们能在正空间法
的全局最优化步骤中高效地对晶胞中的成键关系进行检测,从而可以实时地排除存在
原子重叠的晶体模型,以及实时地进行关于配位多面体、原子键价等等的计算。
在此基础之上值得进一步探索的课题主要是上述晶体学碰撞检测算法框架在
成键关系大部分已知的结构上的应用,以及细碰撞检测正确性的证明。

基于上述机制,本人开发了 \emph{decryst} 这一套利用正空间法和等效点系组合(EPC)
法从指标化的粉末衍射数据求解晶体结构的软件,其中首次以一种一般性的方式使用
增量计算的思想来提升其计算性能,并且通过一种增量的算法生成 EPC 以降低其内存需求,
此外 \emph{decryst} 中也加入了对并行和分布式计算的支持。通过应用以增量计算为
代表的惰性计算技术,以及并行和分布式计算,\emph{decryst} 有着很高的性能;因为
EPC 互相独立且数量往往很大,\emph{decryst} 对 EPC 任务的并行化将为求解
成键关系总体未知的结构带来前所未有的机遇。在此基础之上值得进一步探索的
课题主要是更加复杂的目标函数在 \emph{decryst} 中的应用,以及对
\emph{decryst} 中全局最优化算法的进一步改进。

\emph{decryst} 的设计追求简洁、灵活,而本人也希望其中的技巧可以在更多的晶体学
软件中得到应用;在现有自动化工具的配合下,用 \emph{decryst} 能简单地实现相当
复杂的求解流程。本人以从美国矿物学家晶体结构数据库(AMCSD)中若干个不同晶系和
复杂度的测试结构为例,演示了 \emph{decryst} 的基本用法和常用技巧。在目前工作的
基础之上,主要值得进一步探索的课题是利用增量 EPC 生成算法中剪枝的思路动态地排除
不可能生成无碰撞晶体模型的 EPC,以及对等价 EPC 和等价晶体模型的自动排除。

\begin{rquote}{0.4\textwidth}
	Finally, the number of Unix installations
	has grown to 10, with more expected.
\end{rquote}
\rightline{--- Ken Thompson and Dennis Ritchie (\cite*{thompson1972})}

% vim:ts=4:sw=4
