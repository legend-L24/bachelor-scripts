% Copyright (c) 2014,2016,2018 Casper Ti. Vector
% Public domain.

\chapter{致谢}

首先要感谢的是董成老师,因为如果没有他提出原子重叠的自动化处理这样一个极其适合我
的课题,我很难如此顺利地进行研究生期间的科研工作并完成本文。不仅如此,尽管
我和董成老师之间在不少技术问题上有明显的分歧,他不但没有阻止我投稿,而且主动要求
在我的论文\parencite{liu2017, liu2018}中不署名,甚至批准了我的提前答辩申请;
能这样做的导师在研究生导师中未必是多数,因此我衷心地感谢董成老师的高风亮节。

我也要感谢北大化学院的王颖霞老师和中科院高能所的王焕华老师,因为他们不仅分别
从化学家和物理学家的角度让我入了晶体学的门,而且在晶体学之外的不少方面给了我
很多帮助。还要感谢的是北大数学院的范后宏老师以及 \emph{Tensor Geometry}%
\parencite{dodson1991}的作者,因为前者开设的线性代数课程和后者
关于同一主题的阐述使我对代数和几何之间关系的理解有了质的飞跃,
而数形结合的思路正是解决原子重叠问题的数学基础。

感谢 Ken Thompson、Dennis Ritchie 等等 Unix 先驱以及 Daniel J.\ Bernstein、%
Laurent Bercot 等等 Unix 精神继承者,因为他们所提倡的 Unix 哲学“do one thing
and do it well”\parencite{salus1994}使我不仅能以很高的效率完成 \emph{decryst}
和其它的项目中的编程任务,而且能以更深刻的眼光来审视学习和生活中的问题,
这样的影响是无价的。我也要感谢北大 Linux 俱乐部和其它一些开源社区,
因为它们提供的平台使我接触了来自各家的技术观点,这也是无价的。

感谢 \emph{decryst} 的测试用户,\emph{decryst} 的文档质量在他们的意见和建议之下
得到了极大的提升;感谢 \TeX{}$\big/$\LaTeX{} 社区的许多人,他们的工作极大地
简化了本文的撰写。最后也要感谢 AMCSD 数据库\parencite{downs2003}%
和 Bilbao 服务器\parencite{aroyo2006}的作者在开放获取的精神之下
提供了宝贵的数据和一些有用的工具,以及感谢 \emph{GNU Parallel}%
\parencite{tange2011}作者的工作极大地方便了本文中一些数据的处理。
本文中的相关工作受到了国家自然科学基金(批准号:21271183)
和国家重点研发计划(批准号:2017YFA0302903)的资助。

\begin{rquote}{0.2\textwidth}
	我用一种悲悯的心情来写剧中人物的争执。我诚恳地祈望着
	看戏的人们也以一种悲悯的眼来俯视这群地上的人们。
\end{rquote}
\rightline{——《\emph{〈雷雨〉序}》\parencite{cao1936}}

% vim:ts=4:sw=4
